\makeatletter

% Pour hevea
\newif\ifouthtml\outhtmlfalse
\newcommand{\cutname}[1]{}
% Notations pour les metavariables
\def\var#1{{\it#1}}
\def\nth#1#2{${\it#1}_{#2}$}
\def\nmth#1#2#3{${\it#1}_{#2}^{#3}$}
\def\optvar#1{\textrm{[}\var{#1}\/\textrm{]}}
\def\event{$\bowtie$}
\def\fromoneto#1#2{$#1 = 1, \ldots, #2$}


% Redefining sections macros to make label mandatory
\let\@oldsection=\section
\let\@oldsubsection=\subsection
\let\@oldsubsubsection=\subsection

\newcommand{\ocamldocinputstart}{
\let\section=\@oldsection
\let\subsection=\@oldsubsection
\let\subsubsection=\@oldsubsubsection
}

\renewcommand{\section}{\@ifstar{\@lsectionstar}{\@lsection}}
\renewcommand{\subsection}{\@ifstar{\@lsubsectionstar}{\@lsubsection}}
\renewcommand{\subsubsection}{\@ifstar{\@lsubsubsectionstar}{\@lsubsubsection}}

\newcommand{\@lsection}[2]{\@oldsection{\label{#1}#2}}
\newcommand{\@lsectionstar}[2]{\@oldsection*{\label{#1}#2}}
\newcommand{\@lsubsection}[2]{\@oldsubsection{\label{#1}#2}}
\newcommand{\@lsubsectionstar}[2]{\@oldsubsection*{\label{#1}#2}}
\newcommand{\@lsubsubsection}[2]{\@oldsubsubsection{\label{#1}#2}}
\newcommand{\@lsubsubsectionstar}[2]{\@oldsubsubsection*{\label{#1}#2}}

\newcommand{\lparagraph}[1]{\paragraph{\label{#1}#1}}

% Numerotation
\setcounter{secnumdepth}{2}     % Pour numeroter les \subsection
\setcounter{tocdepth}{1}        % Pour ne pas mettre les \subsection
                                % dans la table des matieres


\def\ttstretch{\tt\spaceskip=5.77pt plus 1.83pt minus 1.22pt}
% La fonte cmr10 a normalement des espaces de 5.25pt non extensibles.
% En 11 pt ca fait 5.77 pt. On lui ajoute la meme flexibilite que
% cmr10 agrandie a 11 pt.

% Pour la traduction "xxxx" -> {\machine{xxxx}} faite par texquote2
\def\machine#1{\mbox{\ttstretch{#1}}}

% Pour la traduction "\begin{verbatim}...\end{verbatim}"
%                    -> "\begin{machineenv}...\end{machineenv}"
% faite aussi par texquote2.
\newenvironment{machineenv}{\alltt}{\endalltt}

% Environnements

\newlength{\versionwidth}
\setbox0=\hbox{\bf Windows:} \setlength{\versionwidth}{\wd0}

\def\versionspecific#1{
  \begin{description}\item[#1:]~\\}

\def\unix{\versionspecific{Unix}}
\def\endunix{\end{description}}
\def\windows{\versionspecific{Windows}}
\def\endwindows{\end{description}}

\def\requirements{\trivlist \item[\hskip\labelsep {\bf Requirements.}]}
\def\endrequirements{\endtrivlist}
\def\installation{\trivlist \item[\hskip\labelsep {\bf Installation.}]}
\def\endinstallation{\endtrivlist}
\def\troubleshooting{\trivlist \item[\hskip\labelsep {\bf Troubleshooting.}]}
\def\endtroubleshooting{\endtrivlist}

\newtheorem{gcrule}{Rule}

% Pour les tables de priorites et autres tableaux a deux colonnes, encadres

\def\tableau#1#2#3{%
\begin{center}
\begin{tabular}{#1}
\hline
#2 & #3 \\
\hline
}
\def\endtableau{\hline\end{tabular}\end{center}}
\def\entree#1#2{#1 & #2 \\}

% L'environnement option

\def\optionitem[#1]{\if@noparitem \@donoparitem
  \else \if@inlabel \indent \par \fi
         \ifhmode \unskip\unskip \par \fi
         \if@newlist \if@nobreak \@nbitem \else
                        \addpenalty\@beginparpenalty
                        \addvspace\@topsep \addvspace{-\parskip}\fi
           \else \addpenalty\@itempenalty \addvspace\itemsep
          \fi
    \global\@inlabeltrue
\fi
\everypar{\global\@minipagefalse\global\@newlistfalse
          \if@inlabel\global\@inlabelfalse \hskip -\parindent \box\@labels
             \penalty\z@ \fi
          \everypar{}}\global\@nobreakfalse
\if@noitemarg \@noitemargfalse \if@nmbrlist \refstepcounter{\@listctr}\fi \fi
\setbox\@tempboxa\hbox{\makelabel{#1}}%
\global\setbox\@labels
\ifdim \wd\@tempboxa >\labelwidth
 \hbox{\unhbox\@labels
       \hskip -\leftmargin
       \box\@tempboxa}\hfil\break
 \else
 \hbox{\unhbox\@labels
       \hskip -\leftmargin
       \hbox to\leftmargin {\makelabel{#1}\hfil}}
 \fi
 \ignorespaces}

\def\optionlabel#1{\bf #1}
\def\options{\list{}{\let\makelabel\optionlabel\let\@item\optionitem}}
\def\endoptions{\endlist}

% L'environnement library (pour composer les descriptions des modules
% de bibliotheque).

\def\comment{\penalty200\list{}{}\item[]}
\def\endcomment{\endlist\penalty-100}

\def\library{
\begingroup
\raggedright
\let\@savedlistI=\@listI%
\def\@listI{\leftmargin\leftmargini\parsep 0pt plus 1pt\topsep 0pt plus 2pt}%
\itemsep 0pt
\topsep 0pt plus 2pt
\partopsep 0pt
}

\def\endlibrary{
\endgroup
}

\def\restoreindent{\begingroup\let\@listI=\@savedlistI}
\def\endrestoreindent{\endgroup}

% ^^A...^^A: compose l'interieur en \tt, comme \verb

\catcode`\^^A=\active
\def{%
\begingroup\catcode``=13\@noligs\ttstretch\let\do\@makeother\dospecials%
\def\@xobeysp{\leavevmode\penalty100\ }%
\@vobeyspaces\frenchspacing\catcode`\^^A=\active\def{\endgroup}}

% Pour l'index

\let\indexentry=\index
\def\index{\indexentry{\jobname}}
\def\ikwd{\indexentry{\jobname.kwd}}

% Les en-tetes personnalises

\pagestyle{myheadings}
\def\partmark#1{\markboth{Part \thepart. \ #1}{}}
\def\chaptermark#1{\markright{Chapter \thechapter. \ #1}}

% nth

\def\th{^{\hbox{\scriptsize th}}}

% Pour annuler l'espacement vertical qui suit un "verbatim"
\def\cancelverbatim{\vspace{-\topsep}\vspace{-\parskip}}% exact.

% Pour annuler l'espacement vertical entre deux \item consecutifs dans \options
\def\cancelitemspace{\vspace{-8mm}}% determine empiriquement

% Pour faire la cesure apres _ dans les identificateurs
\def\={\discretionary{}{}{}}
\def\cuthere{\discretionary{}{}{}}

% Pour la coupure en petits documents

\let\mysection=\section

%%% Augmenter l'espace entre numero de section
%   et nom de section dans la table des matieres.

\def\l@section{\@dottedtocline{1}{1.5em}{2.8em}}  % D'origine: 2.3

% Pour alltt

\def\rminalltt#1{{\rm #1}}

% redefinition de l'environnement alltt pour que les {} \ et % soient
% dans la bonne fonte

\let\@oldalltt=\alltt
\let\@oldendalltt=\endalltt
\renewenvironment{alltt}{%
\begingroup%
\renewcommand{\{}{\char`\{}%
\renewcommand{\}}{\char`\}}%
\renewcommand{\\}{\char`\\}%
\renewcommand{\%}{\char`\%}%
\@oldalltt%
}{%
\@oldendalltt%
\endgroup%
}

% Index stuff -- cf multind.sty

\def\printindex#1#2{\@restonecoltrue\if@twocolumn\@restonecolfalse\fi
  \columnseprule \z@ \columnsep 35pt
  \newpage \phantomsection \twocolumn[{\Large\bf #2 \vskip4ex}]
  \markright{\uppercase{#2}}
  \addcontentsline{toc}{chapter}{#2}
  \@input{#1.ind}}

%%% References to modules in the standard library
\newcommand{\stdmoduleref}[1]{\hyperref[#1]{\texttt{#1}}[\ref{#1}]}

% ocamldoc version
%\newcommand{\docinput}[3]{\input{#1.tex}}

\newcommand{\docitem}[3]{\input{#2.tex}}
\newcommand{\stddocitem}[2]{\input{Stdlib.#1.tex}}
\newenvironment{linklist}{\begingroup\ocamldocinputstart}{\endgroup}



\newenvironment{maintitle}{\begin{center}}{\end{center}}



% Caml-example related command
\newenvironment{camlexample}[1]{
  \ifnum\pdfstrcmp{#1}{toplevel}=0
    \renewcommand{\hash}{\#}
  \else
    \renewcommand{\hash}{}
  \fi
}{}
\newenvironment{caml}{}{}
\newcommand{\ocamlkeyword}{\bfseries}
\newcommand{\ocamlhighlight}{\bfseries\uline}
\newcommand{\ocamlerror}{\bfseries}
\newcommand{\ocamlwarning}{\bfseries}

\definecolor{gray}{gray}{0.5}
\newcommand{\ocamlcomment}{\color{gray}\normalfont\small}
\newcommand{\ocamlstring}{\color{gray}\bfseries}

\newcommand{\?}{\normalsize\tt\hash{} }
\renewcommand{\:}{\small\ttfamily\slshape}

\makeatother
